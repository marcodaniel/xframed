% \iffalse meta-comment
% !TEX program = pdfLaTeX
%<*internal>
\iffalse
%</internal>
%<*readme>
================================================================
================================================================
Working with the command \fbox or \fcolorbox, one has to
handle page breaks by hand. The present package defines the
environment xframed which automatically deals with page breaks.

Author's name: Marco Daniel and Elke Schubert (supports tikz implementation)
License type: lppl

================================================================
The revision history is printed in the documentation.

================================================================
The current development is available at github:
https://github.com/marcodaniel/xframed

================================================================
The package provides 
 * one dtx files,
 * one Makefile (compiling for Linux/Mac),
 * one personal documentclass ltxmdf.cls 


/doc/latex/xframed/
- README.txt
- xframed.pdf


/source/latex/xframed/
- Makefile
- xframed.dtx

/tex/latex/xframed
- xframed.sty
- xltxmdf.cls
================================================================
================================================================
%</readme>
%<*internal>
\fi
\def\nameofplainTeX{plain}
\ifx\fmtname\nameofplainTeX\else
  \expandafter\begingroup
\fi
%</internal>
%<*install>
\input docstrip.tex
\keepsilent
\askforoverwritefalse
\preamble
----------------------------------------------------------------
Working with the command fbox or fcolorbox, one has to
handle page breaks by hand. The present package defines the
environment xframed which automatically deals with page breaks.

Author's name: Marco Daniel and Elke Schubert (!new)
License type: lppl

==================================================
========Is based on the idea of framed.sty========
==================================================
===== Currently the package has a beta-Status ====
==================================================

 Copyright (c) 2010 Marco Daniel

 This package may be distributed under the terms of the LaTeX Project
 Public License, as described in lppl.txt in the base LaTeX distribution.
 Either version 1.0 or, at your option, any later version.


=================================================
 Erstellung eines Rahmens, der am Seitenende keine
 horizontale Linie einfuegt
>>>>>>>>>>>>>>>>>>>>>>>>>>>>>>>>>>>>>>>>>>>>>>>>>
      _______________                            
      |    page 1   |                              
      |    Text     |                            
      |  __Text__   |                            
      |  | Text |   |                            
     P A G E B R E A K                           
      |  | Text |   |                            
      |  |_Text_|   |                            
      |    Text     |                            
      |____page 2___|                            
                                                 
>>>>>>>>>>>>>>>>>>>>>>>>>>>>>>>>>>>>>>>>>>>>>>>>>
==================================================

\endpreamble
\postamble

================================================================
Copyright (C) 2012 by Marco Daniel

This work may be distributed and/or modified under the
conditions of the LaTeX Project Public License (LPPL), either
version 1.3c of this license or (at your option) any later
version.  The latest version of this license is in the file:

http://www.latex-project.org/lppl.txt

This work is "maintained" (as per LPPL maintenance status) by
Marco Daniel.

Have fun!

================================================================
\endpostamble
\usedir{tex/latex/xframed}
\generate{\file{xframed.sty}{\from{xframed.dtx}{package}}}
\Msg{*********************************************************}
\Msg{*}
\Msg{* To finish the installation you have to move the}
\Msg{* following file into a directory searched by TeX:}
\Msg{*}
\Msg{* \space\space documentation.sty}
\Msg{*}
\Msg{* To produce the documentation run the file xframed.dtx}
\Msg{* once through LaTeX. Then, run}
\Msg{*}
\Msg{* \space\space makeindex -s gglo.ist -o xframed.gls xframed.glo}
\Msg{* \space\space makeindex -s gind.ist xframed.idx}
\Msg{*}
\Msg{* through makeIndex to produce the glossary.}
\Msg{* Finally, run PdfLaTeX once again.}
\Msg{*}
\Msg{* To create the examples run (pdf)latex on the tex-files.}
\Msg{*}
\Msg{* That's all!}
\Msg{*}
\Msg{* Happy TeXing!}
\Msg{*********************************************************}
%</install>
%<install>\endbatchfile
%<*internal>
\usedir{source/latex/xframed}
\generate{\file{xframed.ins}
          {\from{xframed.dtx}{install}}}

\nopreamble\nopostamble\usedir{doc/latex/xframed}
\generate{\file{README.txt}
          {\from{xframed.dtx}{readme}}}

\ifx\fmtname\nameofplainTeX
  \expandafter\endbatchfile
\else
  \expandafter\endgroup
\fi
%</internal>
%<*driver>
%%$Id: xframed.dtx 4 2012-05-27 18:13:09Z marco $
\setcounter{errorcontextlines}{999}
\documentclass[parskip=false,11pt,]{xltxmdf}
%%%\usepackage[framemethod=TikZ]{xframed}
\usepackage{xparse}
\GetIdInfo$Id: xframed.dtx 4 2012-05-27 18:13:09Z marco $
          {documentation of xframed}

\EnableCrossrefs
\CodelineIndex
%%\RecordChanges
\begin{document}
  \DocInput{xframed.dtx}
\end{document}
%</driver>
% \fi
%
%
% \CharacterTable
%  {Upper-case    \A\B\C\D\E\F\G\H\I\J\K\L\M\N\O\P\Q\R\S\T\U\V\W\X\Y\Z
%   Lower-case    \a\b\c\d\e\f\g\h\i\j\k\l\m\n\o\p\q\r\s\t\u\v\w\x\y\z
%   Digits        \0\1\2\3\4\5\6\7\8\9
%   Exclamation   \!     Double quote  \"     Hash (number) \#
%   Dollar        \$     Percent       \%     Ampersand     \&
%   Acute accent  \'     Left paren    \(     Right paren   \)
%   Asterisk      \*     Plus          \+     Comma         \,
%   Minus         \-     Point         \.     Solidus       \/
%   Colon         \:     Semicolon     \;     Less than     \<
%   Equals        \=     Greater than  \>     Question mark \?
%   Commercial at \@     Left bracket  \[     Backslash     \\
%   Right bracket \]     Circumflex    \^     Underscore    \_
%   Grave accent  \`     Left brace    \{     Vertical bar  \|
%   Right brace   \}     Tilde         \~}
%
% \GetFileInfo{xframed.sty}
%
%
% \title{The \Pack{xframed} package
%       \footnote{Extending the package \texttt{framed.sty}}}
% \subtitle{auto-split frame environment}
% \author{\href{mailto:marco.daniel@mada-nada.de}{Marco Daniel}%
%        \href{mailto:marco.daniel@mada-nada.de}{Elke Schubert}}
% \version{0.01 ALPHA}
% \GetIdInfo$Id: xframed.dtx 4 2012-05-27 18:13:09Z marco $
%           {documentation of xframed}
% \date{\ExplFileDate}
% \introduction{The standard methods for framing text (\Cmd{fbox}
%              or \Cmd{fcolorbox}) require you to handle page breaks
%              by hand, meaning that you have to split the \Cmd{fbox}
%              into two. The present package defines the environment 
%              \Pack{xframed} which automatically deals with pagebreaks
%              in framed text.\par
%              By defining new environments the user may choose between
%              several individual designs.%
%             \par\kern.5\baselineskip\noindent
%             \par\kern.5\baselineskip\noindent
%             FYI:\quad\parbox[t]{.8\linewidth}{%
%                I create a repository for \Pack{xframed}
%                on \href{https://github.com/marcodaniel/xframed}{github}
%                where you can 
%                \href{https://github.com/marcodaniel/xframed}{download}
%                the current development status.}%
% }
%
% \maketitle
%
% \vspace*{2\baselineskip} 
%
% \section{Motivation}
% \clearpage
%
% \section{Implementation}\label{implementation}
%
% And finally, here's how it all works\ldots
%
%\StopEventually{^^A
%  %%\clearpage
%  %%\PrintChanges^^A
%  \clearpage
%  %\PrintIndex^^A
%}
%\iffalse
%    \begin{macrocode}
%<*package>
%    \end{macrocode}
%\fi
%
% \subsection{The Explanation of xframed.sty}
%
% %$Id: xframed.dtx 4 2012-05-27 18:13:09Z marco $
% %$Rev: 4 $
% %$Author: marco $
% %$Date: 2012-05-27 20:13:09 +0200 (So, 27 Mai 2012) $
% \begin{macro}{mdversion,xframedpackagename}
% Set package information and start header with expl3
%    \begin{macrocode}
\def\mdversion{v2.0}
\def\xframedpackagename{xframed}
\RequirePackage{expl3}
\GetIdInfo$Id: xframed.dtx 4 2012-05-27 18:13:09Z marco $
          {package xframed}

\ProvidesExplPackage{\ExplFileName}
     {\ExplFileDate}{\ExplFileVersion}{\ExplFileDescription}
%    \end{macrocode}
% \end{macro}
%
%
% \begin{macro}{a}
%  Test whether a newer version of \Pack{expl3} is available.
%    \begin{macrocode}
\@ifpackagelater { expl3 } { 2011/09/05 }
  { }
  {
    \PackageError { xframed } { Support~package~expl3~too~old. }
      {
        You~need~to~update~your~installation~of~the~bundles~'l3kernel'~and~
        'l3packages'.\\
        Loading~xframed~will~abort!
      }
    \tex_endinput:D
  }
%    \end{macrocode}
% \end{macro}
%
%
% \begin{macro}{}
%  Loading required packages
%    \begin{macrocode}
\msg_new:nnnn { xframed } { package-not-available }
  { Package~'#1'~is~not~available. }
  { The~#1~package~is~not available~but~xframed~needs~the~package~.\\
    For~further~information~see~the~documenation. }

\cs_new_protected:Npn \xframed_load_check:n #1 {
    \IfFileExists {#1.sty}
      { \RequirePackage{#1} }
      { \msg_error:nnx { xframed } { package-not-available } {#1} }
}

\clist_map_function:nN
  { etoolbox , zref-abspage , xparse , l3keys2e }
    \xframed_load_check:n

%    \end{macrocode}
% \end{macro}
%
% \begin{macro}{}
%  Definining the global option \Opt{framemethod}.
%    \begin{macrocode}
\tl_new:N \xframed_framemethode_tl
\cs_new_protected:Npn \xframed_set_framemethod:n #1 
 {
   \clist_if_in:nnT { default , tex , latex , none , 0 }
                    { \tl_expandable_uppercase:n #1 }
                    { \tl_gset \xframed_framemethode_tl { default } }
   \clist_if_in:nnT { pgf , tikz , 1 } 
                    { \tl_expandable_uppercase:n #1 }
                    { \tl_gset \xframed_framemethode_tl { tikz } }
   \clist_if_in:nnT { pstricks , ps , postscript , 2 , 3 }
                    { \tl_expandable_uppercase:n #1 }
                    { \tl_gset \xframed_framemethode_tl { pstricks } }
 }

\keys_define:nn { xframed }
 {
    framemethod .code:n  = \xframed_set_framemethod:n { #1 }  
 }

%    \end{macrocode}
% \end{macro}
%
%
% \begin{macro}{}
% Command to define a new length option with a default value. 
%    \begin{macrocode}
\prop_new:N \l_xframed_lengthoption_prop

\cs_new_protected:Npn \xframed_prop_set:Nnn #1 #2 #3
 {
  \prop_del:Nn  #1 { #2 } 
  \prop_put:Nnn #1 { #2 } { #3 }
 }

%    \end{macrocode}
% \end{macro}
%
%
% \begin{macro}{}
%    \begin{macrocode}
\cs_new_protected:Npn \xframed_lengthkeys_define:n #1
 {
   \xframed_lengthkeys_define_aux:nn #1 \q_stop
 }

\cs_new_protected:Npn \xframed_lengthkeys_define_aux:nn #1==#2 \q_stop 
 {
   \keys_define:nn { xframed }
    {
      #1 .code:n  =  \xframed_prop_set:Nnn \l_xframed_lengthoption_prop 
                                  { #1 } { \dim_eval:n { ##1 } }
    }
  \keys_set:nn  { xframed }
   {
     #1 = { #2 }
   }
 }

%    \end{macrocode}
% \end{macro}
%
%
% \begin{macro}{}
% Here the declaration of all length options.
%    \begin{macrocode}
\clist_map_function:nN
 {
   skip-above              ==  \c_zero_dim      ,
   skip-below              ==  \c_zero_dim      ,
   left-margin             ==  \c_zero_dim      ,
   right-margin            ==  \c_zero_dim      ,
   inner-left-margin       ==  10pt             ,
   inner-right-margin      ==  10pt             ,
   inner-top-margin        ==  0.4\baselineskip ,
   inner-bottom-margin     ==  0.4\baselineskip ,
   split-topskip           ==  \c_zero_dim      ,
   split-bottomskip        ==  \c_zero_dim      ,
   outer-margin            ==  \c_zero_dim      ,
   inner-margin            ==  \c_zero_dim      ,
   line-width              ==  0.4pt            ,
   inner-line-width        ==  \c_zero_dim      ,
   middle-line-width       ==  .4pt             ,
   outer-line-width        ==  \c_zero_dim      ,
   round-corner            ==  \c_zero_dim      ,
   footenote-distance      ==  \medskipamount   ,
   text-width              ==  \linewidth       ,
   title-above-skip        ==  5pt              ,
   title-below-skip        ==  5pt              ,
   subtitle-above-skip     ==  5pt              ,
   subtitle-below-skip     ==  5pt              ,
   subsubtitle-above-skip  ==  5pt              ,
   subsubtitle-below-skip  ==  5pt              ,
   title-line-width        ==  .2pt             ,
   title-left-margin       ==  10pt             ,
   title-right-margin      ==  10pt             ,
   shadow-size             ==  2pt              ,
   extra-skip-above        ==  \c_zero_dim      ,
   null-skip               ==  \c_zero_dim      ,
 } \xframed_lengthkeys_define:n

%    \end{macrocode}
% \end{macro}
%
%
% \begin{macro}{}
% Command to define a new string option with a default value. 
%    \begin{macrocode}
\prop_new:N \l_xframed_coloroption_prop

\cs_new_protected:Npn \xframed_colorkeys_define:n #1
 {
   \xframed_colorkeys_define_aux:nn #1 \q_stop
 }

\cs_new_protected:Npn \xframed_colorkeys_define_aux:nn #1==#2 \q_stop 
 {
   \keys_define:nn { xframed }
    {
      #1 .code:n  =  \xframed_prop_set:Nnn \l_xframed_coloroption_prop 
                                  { #1 } { ##1 }
    }
  \keys_set:nn  { xframed }
   {
     #1 = { #2 }
   }
 }

%    \end{macrocode}
% \end{macro}
%
% \begin{macro}{}
% Here the declaration of all color options.
%    \begin{macrocode}
\clist_map_function:nN
 {
   line-color              ==  black            ,
   bg-color                ==  white            ,
   font-color              ==  black            ,
   inner-line-color        ==  white            ,
   outer-line-color        ==  white            ,
   middle-line-color       ==  white            ,
   title-font-color        ==  black            ,
   title-line-color        ==  black            ,
   title-bg-color          ==  white            ,
   subtitle-font-color     ==  black            ,
   subtitle-line-color     ==  black            ,
   subtitle-bg-color       ==  white            ,
   subsubtitle-font-color  ==  black            ,
   subsubtitle-line-color  ==  black            ,
   subsubtitle-bg-color    ==  white            ,
   shadow-color            ==  black!50         ,
 } \xframed_colorkeys_define:n

%    \end{macrocode}
% \end{macro}
%
%
% \begin{macro}{}
% Command to define a all bool option with a default value. 
%    \begin{macrocode}
\keys_define:nn { xframed }
 {
  no-ntheorem-preskip  .bool_set:N  = \l_xframed_ntheoremskip_bool     ,
  top-line             .bool_set:N  = \l_xframed_topline_bool          ,
  left-line            .bool_set:N  = \l_xframed_leftline_bool         ,
  bottom-line          .bool_set:N  = \l_xframed_bottomline_bool       ,
  right-line           .bool_set:N  = \l_xframed_rightline_bool        ,
  hide-all-lines       .meta:n      = {
                                       top-line    = #1  ,
                                       left-line   = #1  ,
                                       bottom-line = #1  ,
                                       right-line  = #1  ,
                                      }                                ,
  hide-all-lines        .default:n  = true                             ,
  title-line            .bool_set:N = \l_xframed_titleline_bool        ,
  subtitle-line         .bool_set:N = \l_xframed_subtitleline_bool     ,
  subsubtitle-line      .bool_set:N = \l_xframed_subsubtitleline_bool  ,
  allow-breaking        .bool_set:N = \l_xframed_allowbreaking         ,
  footnote-inside       .bool_set:N = \l_xframed_footnoteinside_bool   ,
  twoside-mode          .bool_set:N = \l_xframed_usetwoside_bool       ,
  repeat-title          .bool_set:N = \l_xframed_repeattitle_bool      ,
  shadow                .bool_set:N = \l_xframed_shadow_bool           ,
  draw-everyline        .bool_set:N = \l_xframed_everyline_bool        ,
  ignore-last-descender .bool_set:N = \l_xframed_descenders_bool       ,
 }

%    \end{macrocode}
% \end{macro}
%
%
% \begin{macro}{}
% Here the declaration of the string options.
%    \begin{macrocode}
\keys_define:nn { xframed }
 {
  title                .tl_set:N    = \l_xframed_title_tl              ,
  code-before          .tl_set:N    = \l_xframed_codebefore_tl         ,
  code-after           .tl_set:N    = \l_xframed_codeafter_tl          ,
  code-before          .tl_set:N    = \l_xframed_codebefore_tl         ,
  code-single-frame    .tl_set:N    = \l_xframed_code_single_frame_tl  ,
  code-first-frame     .tl_set:N    = \l_xframed_code_first_frame_tl   ,
  code-middle-frame    .tl_set:N    = \l_xframed_code_middle_frame_tl  ,
  code-last-frame      .tl_set:N    = \l_xframed_code_last_frame_tl    ,
 }

%    \end{macrocode}
% \end{macro}
%
%
% \begin{macro}{}
% Here the declaration of the align options.
%    \begin{macrocode}
\tl_new:N \l_xframed_alignment_tl
\tl_new:N \l_xframed_titlealignment_tl
\tl_new:N \l_xframed_subalignment_tl
\tl_new:N \l_xframed_subsubalignment_tl

\keys_define:nn { xframed }
 {
%
  alignment             .choice:     ,
   alignment              / left    .code:n = 
      { \tl_set:Nn \l_xframed_alignment_tl { left } }                  ,
   alignment              / right   .code:n = 
      { \tl_set:Nn \l_xframed_alignment_tl { right } }                 ,
   alignment              / center  .code:n = 
      { \tl_set:Nn \l_xframed_alignment_tl { center } }                ,
%
  alignment-title       .choice:     ,
   alignment-title        / left    .code:n = 
      { \tl_set:Nn \l_xframed_titlealignment_tl { left } }             ,
   alignment-title        / right   .code:n = 
      { \tl_set:Nn \l_xframed_titlealignment_tl { right } }            ,
   alignment-title        / center  .code:n = 
      { \tl_set:Nn \l_xframed_titlealignment_tl { center } }           ,
   alignment-title        / center  .code:n = 
      { \tl_set:Nn \l_xframed_titlealignment_tl { justified } }        ,
%
  alignment-subtitle    .choice:     ,
   alignment-title        / left    .code:n = 
      { \tl_set:Nn \l_xframed_subtitlealignment_tl { left } }          ,
   alignment-title        / right   .code:n = 
      { \tl_set:Nn \l_xframed_subtitlealignment_tl { right } }         ,
   alignment-title        / center  .code:n = 
      { \tl_set:Nn \l_xframed_subtitlealignment_tl { center } }        ,
   alignment-title        / center  .code:n = 
      { \tl_set:Nn \l_xframed_subtitlealignment_tl { justified } }     ,
%
  alignmentsubsubtitle  .choice:     ,
   alignment-title        / left    .code:n = 
      { \tl_set:Nn \l_xframed_subsubtitlealignment_tl { left } }       ,
   alignment-title        / right   .code:n = 
      { \tl_set:Nn \l_xframed_subsubtitlealignment_tl { right } }      ,
   alignment-title        / center  .code:n = 
      { \tl_set:Nn \l_xframed_subsubtitlealignment_tl { center } }     ,
   alignment-title        / center  .code:n = 
      { \tl_set:Nn \l_xframed_subsubtitlealignment_tl { justified } }  ,
%
 }

%    \end{macrocode}
% \end{macro}
%
%
%
% \begin{macro}{}
% Option to pass options to tikz or pstricks
%    \begin{macrocode}
\keys_define:nn { xframed }
 {
  post-tikz-code  .tl_set:N = \l_xframed_extratikz_tl                   ,
  setup-tikz      .code:n   = { \tikzset{tikzsetting/.style = {#1} } }  ,
 }
%    \end{macrocode}
% \end{macro}
%
%
% \begin{macro}{}
% Defining the option needspace MUST BE CHANGED -- LATER
%    \begin{macrocode}

%%  \define@key{mdf}{needspace}[\z@]{%
%%       \begingroup%
%%          \setlength{\dimen@}{#1}%
%%          \vskip\z@\@plus\dimen@%
%%          \penalty -100\vskip\z@\@plus -\dimen@%
%%          \vskip\dimen@%
%%          \penalty 9999%
%%          \vskip -\dimen@%
%%          \vskip\z@skip % hide the previous |\vskip| from |\addvspace|
%%        \endgroup%
%%  }

%    \end{macrocode}
% \end{macro}
%
%
% \begin{macro}{}
%    \begin{macrocode}
\msg_new:nnnn { xframed } { unknown-option }
  { Unknown~option~'#1'. }
  {
    The~option~file~'#1'~is~not~known~by~xframed:
    perhaps~it~is~spelled~incorrectly.
  }

\keys_define:nn { xframed }
 {
  unknown .code:n =
        {
         \msg_error:nnx { xframed } { unknown-option }
                        { \exp_not:V \l_keys_key_tl }
        }
 }

%    \end{macrocode}
% \end{macro}
%
%
% \begin{macro}{}
%    \begin{macrocode}
\ProcessKeysOptions { xframed }

%    \end{macrocode}
% \end{macro}
%
%
% \begin{macro}{xframedsetup}
%    \begin{macrocode}
\NewDocumentCommand \xframedsetup { m }
 {
 \keys_set:nn { xframed } { #1 }
 }

%    \end{macrocode}
% \end{macro}
%
%
% \begin{macro}{style}
%    \begin{macrocode}
\msg_new:nnnn { xframed } { unknown-style }
  { Unknown~style~'#1'. }
  {
    The~style~'#1'~is~not~known~by~xframed:
    perhaps~it~is~spelled~incorrectly.
  }
\keys_define:nn { xframed }
 {
  style .code:n =
    {
     \cs_if_exist:cTF { xframed_stylename_#1 }
      {
       \xframedsetup { \use:c { xframed_stylename_#1 } }
      }
      {
       \msg_error:nnx { xframed } { unknown-style }
                      { \exp_not:V \l_keys_key_tl }
      }
    }
 }

%    \end{macrocode}
% \end{macro}
%
%
% \begin{macro}{}
%  Initialize all commands and length which will we used later
%    \begin{macrocode}
\coffin_new:N \l_xframed_store_one_coffin
\coffin_new:N \l_xframed_store_two_coffin
\coffin_new:N \l_xframed_store_save_coffin

%    \end{macrocode}
% \end{macro}
%
%
% \begin{macro}{}
%    \begin{macrocode}
\cs_new:Npn \xframed_calculate_width:
 {
  \dim_set:Nn \l_tmpa_dim { \linewidth }
  \clist_map_inline:nn 
    { 
     left-margin , outer-line-width , middle-line-width , inner-line-width , 
     inner-left-margin , inner-right-margin , inner-line-width , 
     middle-line-width , outer-line-width , right-margin
    }
    {
     \dim_sub:Nn \l_tmpa_dim 
           { 
            \prop_get:Nn \l_xframed_lengthoption_prop { ##1 }
           }
    }
 }

%    \end{macrocode}
% \end{macro}
%
%
% \begin{macro}{}
%    \begin{macrocode}
\NewDocumentEnvironment { xframed } { O{} }
 {
 \group_begin:
   \xframedsetup { #1 }  
   \xframed_trivlist:
    \item\relax
    \xframed_calculate_width: 
    \vcoffin_set:Nnw \l_xframed_store_one_coffin { \l_tmpa_dim }
 }
 {
    \par \tex_unskip:D
    \skip_vertical:N \tex_lastskip:D
    \par
    \hbox:n { 
             \tex_vrule:D width  0   pt 
                          height 0.7 \tex_baselineskip:D 
                          depth  0.3 \tex_baselineskip:D
             \scan_stop: 
            }
    \par \tex_unskip:D \tex_unskip:D
    \setbox0=\lastbox
    \skip_vertical:n { 0.7 \tex_baselineskip:D - \tex_baselineskip:D }
    \vcoffin_set_end:
%    \coffin_rotate:Nn \l_xframed_store_one_coffin { 0 } % must be <0
    \coffin_display_handles:Nn \l_xframed_store_one_coffin { red!70!black }
   \endxframed_trivlist:
  \group_end:
  \@doendpe
 }




\cs_new:Npn \xframed_trivlist:
 {
  \setlength{\topsep}
         { \prop_get:Nn \l_xframed_lengthoption_prop { skip-above } }
  \parsep\parskip
  \@nmbrlistfalse
  \@trivlist
  \labelwidth\z@
  \leftmargin\z@
  \itemindent\z@
  \let\@itemlabel\@empty
  \def\makelabel##1{##1}
 }
\cs_new:Npn \endxframed_trivlist:
 {
  \if@inlabel
    \leavevmode
    \global \@inlabelfalse
  \fi
  \if@newlist
    \@noitemerr
    \global \@newlistfalse
  \fi
  \ifhmode\unskip \par
  \else
    \@inmatherr{\end{\@currenvir}}%
  \fi
  \if@noparlist \else
    \ifdim\lastskip >\z@
      \@tempskipa\lastskip \vskip -\lastskip
      \advance\@tempskipa\parskip \advance\@tempskipa -\@outerparskip
      \vskip\@tempskipa
    \fi
    \addpenalty\@endparpenalty
    \addvspace { \prop_get:Nn \l_xframed_lengthoption_prop { skip-below } }
    \@endpetrue
  \fi
 }

%    \end{macrocode}
% \end{macro}
%
%
% \begin{macro}{}
%    \begin{macrocode}
  \tex_endinput:D

%    \end{macrocode}
% \end{macro}
%
%\iffalse
%    \begin{macrocode}
%</package>
%    \end{macrocode}
%\fi

